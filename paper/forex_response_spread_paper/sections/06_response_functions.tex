\section{Price response functions}
\label{sec:response_functions}
In Sect. \ref{subsec:key_concepts} we establish the fundamental quantities used
in the price response definitions.
In Sect. \ref{subsec:response_function_trade} we analyze the responses
functions in trade time scale and in Sect. \ref{subsec:response_function_physical}
we analyze the responses functions in physical time scale.

%%%%%%%%%%%%%%%%%%%%%%%%%%%%%%%%%%%%%%%%%%%%%%%%%%%%%%%%%%%%%%%%%%%%%%%%%%%%%%%
\subsection{Key concepts}\label{subsec:key_concepts}

In general, three categories of currency pairs are defined: majors, crosses,
and exotics. The ``major'' foreign exchange currency pairs are the most
frequently traded currencies that are paired with the U.S. dollar (see Table
\ref{tab:majors}). The ``crosses'' are those majors pairs paired between them
and that exclude the U.S. dollar. Finally, the ``exotic'' pairs usually consist
of a major currency alongside a thinly traded currency or an emerging market
economy currency. The majors are the most liquid pairs, in contrast with the
exotics, who can be much more volatile.

In foreign exchange markets, orders are execute at the best available buy or
sell price. Orders often fail to result in an immediate transaction, and are
stored in a queue called the limit order book
\cite{forex_market_micro,forex_structure,stat_prop,predictive_pow,intro_market_micro,prop_order_book}.
The order book is visible for all traders and its main purpose is to ensure
that all traders have the same information on what is offered on the market.
For a detailed description of the operation of the markets, we suggest to see
Ref. \cite{my_paper_response_financial_arxiv}.

At any given time there is a best (lowest) offer to sell with price
$a\left(t\right)$, and a best (highest) bid to buy with price $b\left(t\right)$
\cite{subtle_nature,prop_order_book,account_spread,limit_ord_spread,stat_theory}.
The price gap between them is called the spread
$s\left(t\right) = a\left(t\right)-b\left(t\right)$
\cite{teach_spread,subtle_nature,Bouchaud_2004,large_prices_changes,market_digest,account_spread,stat_theory,em_stylized_facts}.
Spreads are significantly positively related to price and significantly
negatively related to trading volume. Companies with more liquidity tend to
have lower spreads
\cite{components_spread_tokyo,account_spread,effects_spread,components_spread}.
Despite the foreign exchange market is often cited as the world's largest
financial market, this description fail to consider the considerable differences
in trading volume and liquidity across different currency pairs
\cite{forex_microstructure}. These differences can be directly seen in the
spread. Furthermore, the bid-ask spread is directly related with the
transaction costs to the dealer \cite{teach_spread,spread_futures}.

As there is no price information in the data, we consider a basic definition of
the price given by \cite{forex_liquidity,patterns_forex,political_forex}. The
average of the best ask and the best bid is the midpoint price, which is
defined as
\cite{teach_spread,subtle_nature,Bouchaud_2004,my_paper_response_financial_arxiv,large_prices_changes,prop_order_book,stat_theory,em_stylized_facts}
\begin{equation}
    m \left(t\right) = \frac{a\left(t\right) + b\left(t\right)}{2}.
\end{equation}
Price changes are typically characterized as returns. If one denotes
$S\left( t\right)$ the price of an asset at time $t$, the return
$r^{\left(g\right)}\left(t, \tau\right)$, at time $t$ and time lag $\tau$ is
simply the relative variation of the price from $t$ to $t + \tau$
\cite{subtle_nature,empirical_facts,asynchrony_effects_corr,tick_size_impact,causes_epps_effect,non_stationarity},
\begin{equation}\label{eq:return_general}
    r^{\left(g\right)} \left(t, \tau \right) = \frac{S\left(t + \tau\right)
    - S\left(t\right)}{S\left(t\right)}.
\end{equation}
We define the returns via the midpoint price as
\begin{equation}\label{eq:midpoint_price_return}
    r\left(t,\tau\right) = \frac{m\left(t+\tau\right)-m\left(t\right)}
    {m\left(t\right)}.
\end{equation}
The distribution of returns is strongly non-Gaussian and its shape continuously
depends on the return period $\tau$. Small $\tau$ values have fat tails return
distributions \cite{subtle_nature}. The trade signs are defined for general
cases as
\begin{equation}\label{eq:trade_sign_general}
    \varepsilon\left(t\right)=\text{sign}\left(S\left(t\right)
    -m\left(t-\delta\right)\right),
\end{equation}
where $\delta$ is a positive time increment. Hence we have
\begin{equation}\label{eq:trade_sign_results}
    \varepsilon\left(t\right)=\left\{
    \begin{array}{cc}
    +1, & \text{If } S\left(t\right)
    \text{ is higher than the last } m\left( t \right)\\
    -1, & \text{If } S\left(t\right)
    \text{ is lower than the last } m\left( t \right)
    \end{array}\right. .
\end{equation}
Here, $\varepsilon(t) = +1$ indicates that the trade was triggered by a market
order to buy and a trade triggered by a market order to sell yields
$\varepsilon(t) = -1$
\cite{subtle_nature,Bouchaud_2004,spread_changes_affect,quant_stock_price_response,order_flow_persistent}.

The main objective of this work is to analyze the price response functions. In
general we define the price response functions in a foreign exchange market as
\begin{equation}\label{eq:response_general}
    R^{\left(\textrm{scale}\right)}_{ii}\left(\tau\right)=\left\langle
    r^{\left(\textrm{scale}\right)}_{i}\left(t-1, \tau\right)
    \varepsilon^{\left(\textrm{scale}\right)}_{i} \left(t\right)\right\rangle
    _{\textrm{average}},
\end{equation}
where the index $i$ correspond to currency pairs in the market,
$r^{\left(\textrm{scale}\right)}_{i}$ is the return of the pair $i$ in a time
lag $\tau$ in the corresponding scale and
$\varepsilon^{\left(\textrm{scale}\right)}_{i}$ is the trade sign of the pair
$i$ in the corresponding scale. The superscript scale refers to the time scale
used, whether physical time scale ($\textrm{scale} = \textrm{p}$) or trade time
scale ($\textrm{scale} = \textrm{t}$). Finally, The subscript average refers to
the way to average the price response, whether relative to the physical time
scale ($\textrm{average} = P$) or relative to the trade time scale
($\textrm{average} = T$).

For correlated financial markets, the price response function increase to a
maximum and then slowly decrease. This result is observed empirically in trade
time scale and in physical time scale
\cite{my_paper_response_financial_arxiv,Wang_2016_avg}.

%%%%%%%%%%%%%%%%%%%%%%%%%%%%%%%%%%%%%%%%%%%%%%%%%%%%%%%%%%%%%%%%%%%%%%%%%%%%%%%
\subsection{Response functions on trade time scale}
\label{subsec:response_function_trade}

The price response function in trade time scale is defined as
\cite{my_paper_response_financial}
\begin{equation}\label{eq:response_functions_trade_scale_general}
    R^{\left(\textrm{t}\right)}_{ii}\left(\tau\right)=\left\langle
    r^{\left(\textrm{t}\right)}_{i}\left(t-1,\tau \right)
    \varepsilon_{i}^{\left(\textrm{t}\right)}
    \left(t, n\right)\right\rangle _{T}.
\end{equation}
To compute the response functions on trade time scale, we use both, the trade
signs and the returns during a week in market time.

\begin{figure}[htbp]
    \centering
    \includegraphics[width=\columnwidth]
    {figures/04_responses_trade_scale.png}
    \caption{Price response functions
             $R^{\left(\textrm{t}\right)}_{ii}\left(\tau\right)$ versus time
             lag $\tau$ on a logarithmic scale in trade time scale for the
             years 2008 (top), 2014 (middle) and 2019 (bottom).}
    \label{fig:response_function_trade_scale}
\end{figure}

The results of Fig. \ref{fig:response_function_trade_scale} show the
price response functions of the seven foreign exchange major pairs used in the
analysis (see Table \ref{tab:majors}) for three different years. For all the
years it can be seen the same behavior observed in financial correlated
markets. The response functions increase to a maximum and then slowly decrease.
For our selected pairs, a time lag of $\tau = 10^{3}s$ is enough to see an
increase to a maximum followed by a decrease. Thus, the trend in the price
response functions is eventually reversed. The response signal is much more
noisier in the year 2008 for the first seconds in the time lag. This behavior
is because of the smaller amount of data of the corresponding year. In general,
more data was recorded in recent years than in past years. In the three years
analyzed, the more liquid currency pairs have a smaller response in comparison
with the non-liquid pairs. The strength of the response function vary from one
year to the other. In 2008 the strength of the signal is one order of magnitude
stronger than the response in 2014, but the signals in 2014 have approximately
twice the strength the signals of 2019. This behavior can be explained by the
fact that in recent times algorithm trading has been used intensively. Thus,
many more trades were carried out in the last years, which means, the impact of
each trade is reduced, and then the response functions tend to decrease
compared with previous years.

%%%%%%%%%%%%%%%%%%%%%%%%%%%%%%%%%%%%%%%%%%%%%%%%%%%%%%%%%%%%%%%%%%%%%%%%%%%%%%%
\subsection{Response functions on physical time scale}
\label{subsec:response_function_physical}

One important detail to compute the price response function on physical time
scale is to define how the averaging of the function will be made, because the
response functions highly differ when we include or exclude
$\varepsilon^{\left(\textrm{p}\right)}_j \left( t\right) = 0$
\cite{Wang_2016_cross}. The price responses including
$\varepsilon^{\left(\textrm{p}\right)}_j \left( t\right) = 0$ are weaker than
the excluding ones due to the omission of direct influence of the lack of
trades. However, either including or excluding
$\varepsilon^{\left(\textrm{p}\right)}_j \left( t\right) = 0$ does not change
the trend of price reversion versus the time lag, but it does affect the
response function strength \cite{Wang_2016_avg}. For a deeper analysis of the
influence of the term
$\varepsilon^{\left(\textrm{p}\right)}_j \left( t\right) = 0$ in price response
functions, we suggest to review Refs. \cite{Wang_2016_avg,Wang_2016_cross}. We
will only take into account the price response functions excluding
$\varepsilon^{\textrm{p}}_j \left( t\right) = 0$.

We define the price response functions on physical time scale, using
the trade signs and the returns in physical time scale. The price response
function on physical time scale is defined as
\cite{my_paper_response_financial}
\begin{equation}\label{eq:response_functions_time_scale_general}
    R^{\left(\textrm{p}\right)}_{ii}\left(\tau\right)=\left\langle
    r^{\left(\textrm{p}\right)}_{i}\left(t-1, \tau\right)
    \varepsilon_{i}^{\left(\textrm{p}\right)} \left(t\right)\right\rangle _{P}
\end{equation}
\begin{figure}[htbp]
    \centering
    \includegraphics[width=\columnwidth]
    {figures/04_responses_physical_scale.png}
    \caption{Price response functions
             $R^{\left(\textrm{p}\right)}_{ii}\left(\tau\right)$ excluding
             $\varepsilon^{\left(\textrm{p}\right)}_{i}\left(t\right) = 0$ versus time
             lag $\tau$ on a logarithmic scale in physical time scale for the
             years 2008 (top), 2014 (middle) and 2019 (bottom).}
    \label{fig:response_function_physical_scale}
\end{figure}
The results shown in Fig. \ref{fig:response_function_physical_scale} are the
price response functions on physical time scale for three different years. The
results show approximately the same behavior observed in currency exchange
pairs in trade time scale, and in correlated financial markets, where we can
see that an increase to a maximum is followed by a decrease. Thus again, the
trend in the price responses is eventually reversed. An exception occurs in the
year 2008, where the response at short time lags seems to decrease, to then
start to slightly increase, and finally it decrease again.

The price response functions on physical time scale are smoother than the
responses on trade time scale. As we reduce from trade data all the returns and
trade signs in one second to one data point on physical time scale, and as this
sampling gives the same weight to every data point, the curves look smoother.

Compared with the response functions on trade time scale, the strength of the
signal of the response functions on physical time scale are similar in
magnitude in the corresponding years. Thus, the strength of the signal in 2008
for trade time scale is similar to the strength of the signal in 2008 for
physical time scale, and so on. This behavior is different to the one presented
in correlated financial markets, where the results differ about a factor of
two depending on the time scale.

On physical time scale, we can see that the liquid pairs have a smaller price
response compared with non-liquid pairs. Therefore, the price response of a
foreign exchange pair with large activity is smaller to the small impact of
each trade. Also, the older the response, the stronger the signal. We consider
the same argument of algorithm trading to explain why the signals in recent
years are weaker than in older years.
