\section{Time scale}\label{sec:time_scale}

In Sect. \ref{subsec:time_definition} we describe the physical time scale and
the trade time scale. In Sect. \ref{subsec:trade_time} and Sect.
\ref{subsec:physical_time} we define the trade and the physical time scales,
respectively.

%%%%%%%%%%%%%%%%%%%%%%%%%%%%%%%%%%%%%%%%%%%%%%%%%%%%%%%%%%%%%%%%%%%%%%%%%%%%%%%
\subsection{Time definition}\label{subsec:time_definition}

Due to the nature of the data, they are several options to define time for
analyzing data. In general, the time series are labeled in calendar time
(hours, minutes, seconds, milliseconds). Moreover, tick-by-tick data available
on financial markets all over the world is time stamped up to the millisecond,
but the order of magnitude of the guaranteed precision is much larger, usually
one second or a few hundreds of milliseconds
\cite{market_digest,empirical_facts}. In several papers are used different time
definitions (calendar time, physical time, event time, trade time, tick time)
\cite{empirical_facts,sampling_returns,market_making}. The spot foreign
exchange market data used in the analysis only has the quotes. In consequence,
we have to infer the trades during the market time. As we have tick-by-tick
resolution, we can use either trade time scale or physical time scale.

The trade time scale increases by one unit each time a transaction happens,
which in our case is every time the quotes change. The advantage of this count
is that limit orders far away in the order book do not increase the time by one
unit. The main outcome of trade time scale is its ``smoothing" of data and the
aggregational normality \cite{empirical_facts}.

The physical time scale is increased by one unit each time a second passes.
This means that computing the responses in this scale involves sampling
\cite{Wang_2016_cross,sampling_returns}, which has to be done carefully when
dealing for example with several stocks with different liquidity. This sampling
is made in the trade signs and in the midpoint prices.

We use these two definitions of time scale to compute the price response
function. Between these two scales, there is not a judgment which is better or
worse. Their use directly depend on the application. Thus, our aim is to
present how the price response function behaves under these two different time
scales.

%%%%%%%%%%%%%%%%%%%%%%%%%%%%%%%%%%%%%%%%%%%%%%%%%%%%%%%%%%%%%%%%%%%%%%%%%%%%%%%
\subsection{Trade time scale}\label{subsec:trade_time}

As a first approximation, we use the trade sign classification in trade time
scale proposed in Ref. \cite{Wang_2016_cross} and used in Refs.
\cite{my_paper_response_financial,Wang_2016_avg,Wang_2017,Wang_2018_copulas}
that reads
\begin{equation}\label{eq:trade_signs_trade}
    \varepsilon^{\left(\textrm{t}\right)}\left(t,n\right)=\left\{
    \begin{array}{cc}
    \text{sgn}\left(m\left(t,n\right)-m\left(t,n-1\right)\right),
    & \text{if }\\ m\left(t,n\right) \ne m\left(t,n-1\right)\\
    \varepsilon^{\left(\textrm{t}\right)}\left(t,n-1\right),
    & \text{otherwise}
    \end{array}\right..
\end{equation}
Here, $\varepsilon^{\left(\textrm{t}\right)}\left( t,n \right) = +1$ implies a
trade triggered by a market order to buy, and a value
$\varepsilon^{\left(\textrm{t}\right)}\left( t,n \right) = -1$ indicates a trade
triggered by a market order to sell.

In the second case of Eq. (\ref{eq:trade_signs_trade}), if two consecutive
trades with the same trading direction do not exhaust all the available volume
at the best quote, the trades would have the same price, and they will thus
have the same trade sign.

With this classification we obtain trade signs for every single trade in the
data set. According to Ref. \cite{Wang_2016_cross}, the average accuracy of the
classification is $85\%$ for the trade time scale.

%%%%%%%%%%%%%%%%%%%%%%%%%%%%%%%%%%%%%%%%%%%%%%%%%%%%%%%%%%%%%%%%%%%%%%%%%%%%%%%
\subsection{Physical time scale}\label{subsec:physical_time}

We use the trade sign definition in physical time scale proposed in Ref.
\cite{Wang_2016_cross} and used in Refs.
\cite{Wang_2016_avg,Wang_2017}, that depends on the classification in
Eq. (\ref{eq:trade_signs_trade}) and reads
\begin{equation}\label{eq:trade_signs_physical}
    \varepsilon^{\left(\textrm{p}\right)}\left(t\right)=\left\{
    \begin{array}{cc}
    \text{sgn}\left(\sum_{n=1}^{N\left(t\right)}
    \varepsilon^{\left(\textrm{t}\right)} \left(t,n\right)\right),
    & \text{If }N \left(t\right)>0\\
    0, & \text{If }N\left(t\right)=0
    \end{array}\right. ,
\end{equation}
where $N \left(t \right)$ is the number of trades in an interval of one second.
Here, $\varepsilon^{\left(\textrm{p}\right)}\left( t \right) = +1$ implies that
the majority of trades in second $t$ are triggered by a market order to buy,
and a value $\varepsilon^{\left(\textrm{p}\right)}\left( t \right) = -1$
indicates a majority of sell market orders. In this definition, they are two
ways to obtain $\varepsilon^{\left(\textrm{p}\right)}\left( t \right) = 0$.
First, there are not trades in a particular second and thus no trade sign.
Second, the sum of the trade signs in a given second amounts to zero,
indicating an exact balance of buy and sell market orders.

Market orders show opposite trade directions as compared to limit orders
executed simultaneously. An executed sell limit order corresponds to a
buyer-initiated market order. An executed buy limit order corresponds to a
seller-initiated market order. In this case we do not compare every single
trade sign in a second, but the net trade sign obtained for every second with
the definition, see Eq. (\ref{eq:trade_signs_physical}). According to Ref.
\cite{Wang_2016_cross}, this definition has an average accuracy up to $82\%$ in
the physical time scale.
