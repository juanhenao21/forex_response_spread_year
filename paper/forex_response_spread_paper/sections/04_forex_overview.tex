\section{Foreign exchange market overview}\label{sec:forex_overview}

In Sect. \ref{subsec:forex_market} we describe the basic characteristics of the
foreign exchange market. In Sect. \ref{subsec:key_concepts} we establish the
fundamental quantities used in the price response definitions.

%%%%%%%%%%%%%%%%%%%%%%%%%%%%%%%%%%%%%%%%%%%%%%%%%%%%%%%%%%%%%%%%%%%%%%%%%%%%%%%
\subsection{Foreign exchange market}\label{subsec:forex_market}

The foreign exchange market is a 24-hour global decentralized or
over-the-counter (OTC) market for the trading of currencies closing only on the
weekends.
The foreign exchange market is the most volatile, liquid and largest of all
financial markets
\cite{forex_liquidity,info_forex,intraday_forex,forex_structure,teach_spread,forex_market_micro,book_forex,book_forex_2,book_forex_3},
and it has a paramount importance for the world economy. It affects employment,
inflation, international capital flows, among others \cite{forex_structure}.
The major participants trading in this market include governments, central
banks, global funds, retail clients and corporations
\cite{book_forex_2,book_forex_3}. Trading of currency in the foreign exchange
market involves the purchase and sale of two currencies at the same time
\cite{book_forex,book_forex_2,book_forex_3}. The value of one of the currencies
in that pair is relative to the value of the other. The price one currency can
be exchanged with another currency is the foreign exchange rate. The foreign
exchange market is a closed system. As one value increases another value has
to decrease. All foreign exchange rates cannot appreciate, in contrast to the
stock market \cite{book_forex,book_forex_3}.

Depending on the country, the currencies can be ``free float'' or
``fixed float''. Free-floating currencies relative value is determined by
free-market forces. Some example of free-floating currencies include the U.S.
dollar, Japanese yen and Colombian peso. On the other hand, a fixed float is
where a government through the central bank set the currency's relative value
to other currencies, usually by pegging it to some standard. Examples of fixed
floating currencies include the Chinese Yuan and the Indian Rupee
\cite{book_forex}. In our case, we only use free float currencies.

In the foreign exchange market, the trading day begins in Australia and Asia.
Then the markets in Europe open and finally the markets in America
\cite{forex_structure,forex_market_micro,book_forex_2,book_forex_3}. As the
market close time in New York overlaps the market open time in Australia and
Asia, the markets do not formally close during the week. Thus, using the New
York time as reference, the market opens on Sunday at 19h00 and closes on
Friday at 17h00. London, New York and Tokyo are the largest trading centers of
foreign exchange trading \cite{book_forex_4}

\textcolor{red}{Currency markets are divided into spot market, forward, future, currency swaps
and currency options \cite{book_forex_2,book_forex_3,book_forex_4}. The spot and forward
exchange markets are OTC markets \cite{book_forex}. In our work
we particularly focus on the spot market, where as his name suggest, the trades
are settled on the spot \cite{book_forex,book_forex_3}. In a spot market, as
the currency transactions are carried in the OTC markets, information
concerning open interest and volume is unavailable. The transactions in this
market represent up to the 40\% of the total market transactions in the foreign
exchange market. This estimations are made by the Bank for International Settlements (BIS)
based on a central bank survey of foreign exchange and derivatives market activities
in major financial centers \cite{bis}.} The most traded currencies in the spot market are the U.S.
dollar, euro, Japanese yen, British pound and Swiss franc \cite{book_forex}.

In general, three categories of currency pairs are defined: majors, crosses,
and exotics. The ``major'' foreign exchange currency pairs are the most
frequently traded currencies that are paired with the U.S. dollar. The
``crosses'' are those majors pairs paired between them and that exclude the
U.S. dollar. Finally, the ``exotic'' pairs usually consist of a major currency
alongside a thinly traded currency or an emerging market economy currency. The
majors are the most liquid pairs, in contrast with the exotics, who can be much
more volatile. In this work, we will refer as the ``major currency pairs'' to
the pairs of most traded currencies paired with the U.S. dollar, including the
so called commodity currencies: Canadian dollar, Australian dollar and New
Zealand dollar. The pairs and their corresponding symbol can be seen in Table
\ref{tab:majors}.

\begin{table}[htbp]
\centering
\begin{threeparttable}
\caption{Analyzed currency pairs.}
\begin{tabular*}{\columnwidth}{P{5cm}P{3cm}}
\toprule
\bf{Currency pair} & \bf{Symbol} \tabularnewline
\midrule
euro/U.S dollar& EUR/USD \tabularnewline
British pound/U.S. dollar& GBP/USD \tabularnewline
Japanese yen/U.S. dollar& JPY/USD \tabularnewline
Australian dollar/U.S. dollar& AUD/USD \tabularnewline
U.S. dollar/Swiss franc& USD/CHF \tabularnewline
U.S. dollar/Canadian dollar& USD/CAD \tabularnewline
New Zealand dollar/U.S. dollar& NZD/USD \tabularnewline
\bottomrule
\end{tabular*}
\label{tab:majors}
\end{threeparttable}
\end{table}

The term pip (Price Increment Point) is commonly used in the foreign exchange
market instead of tick. The precise definition of a pip is a matter of
convention. Usually, it refers to the incremental value in the fifth non-zero
digit position from the left. It is not related to the position of the decimal
point. For example, one pip in the exchange rate USD/JPY of 124.21 would be
0.01, while one pip for EUR/USD of 1.1021 would be 0.0001
\cite{forex_structure,micro_eff,forex_market_micro,book_forex_3,order_flow_forex}.

Compared with other markets like the stock market, there are some key
characteristics that differentiate the foreign spot exchange market.
\textcolor{red}{There are
fewer rules, there are no clearing houses and central bodies that oversee the
market. Some investors do not have to pay fees or commissions as on other markets.
It is possible to trade at any time of day and regarding the risk and reward,
it is possible to get in and out whenever the investor wants. In the foreign
exchange market, the bid-ask spread is the most used transaction cost
\cite{book_forex_2}}.

%%%%%%%%%%%%%%%%%%%%%%%%%%%%%%%%%%%%%%%%%%%%%%%%%%%%%%%%%%%%%%%%%%%%%%%%%%%%%%%
\subsection{Key concepts}\label{subsec:key_concepts}

In spot foreign exchange markets, orders are executed at the best available buy
or sell price. Orders often fail to result in an immediate transaction, and are
stored in a queue called the limit order book
\cite{forex_structure,forex_market_micro,stat_prop,predictive_pow,intro_market_micro,prop_order_book}.
The order book is visible for all traders and its main purpose is to ensure
that all traders have the same information on what is offered on the market.
For a detailed description of the operation of the markets, we suggest to see
Ref. \cite{my_paper_response_financial}.

At any given time there is a best (lowest) offer to sell with price
$a\left(t\right)$, and a best (highest) bid to buy with price $b\left(t\right)$
\cite{subtle_nature,book_forex,prop_order_book,account_spread,limit_ord_spread,stat_theory}.
The price gap between them is called the bid-ask spread
$s\left(t\right) = a\left(t\right)-b\left(t\right)$
\cite{teach_spread,subtle_nature,Bouchaud_2004,book_forex,account_spread,stat_theory,large_prices_changes,market_digest,em_stylized_facts}.
Bid-ask spreads are significantly positively related to price. Currencies with
more liquidity tend to have lower bid-ask spreads
\cite{components_spread_tokyo,account_spread,effects_spread,components_spread}.
In spot foreign exchange markets, the existing bid-ask spread in any currency
will vary depending on the currency trader, the currency being traded and the
conditions in the market. Although the foreign exchange market is often cited
as the world's largest financial market, this description fails to consider the
considerable differences in trading volume and liquidity across different
currency pairs \cite{forex_microstructure,book_forex_2}. These differences can
be directly seen in the bid-ask spread. The bid-ask spread will tend to
increase for currencies that do not generate a large volume of trading
\cite{book_forex_2}. Furthermore, the bid-ask spread is directly related with
the transaction costs to the dealer
\cite{teach_spread,spread_futures,book_forex_2}.

As we have the quotes prices in the data, we need to infer the trade price. We
consider a basic definition of the price given by
\cite{forex_liquidity,patterns_forex,political_forex}. The average of the best
ask and the best bid is the midpoint price, which is defined as
\cite{teach_spread,subtle_nature,Bouchaud_2004,my_paper_response_financial,prop_order_book,stat_theory,large_prices_changes,em_stylized_facts}
\begin{equation}
    m \left(t\right) = \frac{a\left(t\right) + b\left(t\right)}{2}.
\end{equation}
Price changes are typically characterized as returns. Using the midpoint price
$m\left( t\right)$ of a currency pair at time $t$, the return
$r\left(t, \tau\right)$, at time $t$ and time lag $\tau$ is simply the relative
variation of the price from $t$ to $t + \tau$
\cite{subtle_nature,empirical_facts,asynchrony_effects_corr,tick_size_impact,causes_epps_effect,non_stationarity},
\begin{equation}\label{eq:midpoint_price_return}
    r\left(t,\tau\right) = \frac{m\left(t+\tau\right)-m\left(t\right)}
    {m\left(t\right)}.
\end{equation}
The distribution of returns is strongly non-Gaussian and its shape continuously
depends on the return period $\tau$. Small $\tau$ values have fat tails return
distributions \cite{subtle_nature}. The trade signs are defined for general
cases as
\begin{equation}\label{eq:trade_sign_general}
    \varepsilon\left(t\right)=\text{sign}\left(S\left(t\right)
    -m\left(t-\delta\right)\right),
\end{equation}
\textcolor{red}{where $\delta$ is a positive time increment and $S\left(t\right)$ is the price.}
Hence we have
\begin{equation}\label{eq:trade_sign_results}
    \varepsilon\left(t\right)=\left\{
    \begin{array}{cc}
    +1, & \text{If } S\left(t\right)
    \text{ is higher than the last } m\left( t \right)\\
    -1, & \text{If } S\left(t\right)
    \text{ is lower than the last } m\left( t \right)
    \end{array}\right. .
\end{equation}
Here, $\varepsilon(t) = +1$ indicates that the trade was triggered by a market
order to buy and a trade triggered by a market order to sell yields
$\varepsilon(t) = -1$
\cite{subtle_nature,Bouchaud_2004,spread_changes_affect,quant_stock_price_response,order_flow_persistent}.
In our implementation of the price response functions we need a more specific
definition of trade signs. We give a deeper explanation of trade signs
depending on the time scale in Sect. \ref{sec:time_scale}.

\textcolor{red}{Generally, price response functions measure price changes implied by execution
of market orders and are defined as follows:
\begin{equation}\label{eq:response_general}
    R^{\left(\textrm{scale}\right)}_{i}\left(\tau\right)=\left\langle
    r^{\left(\textrm{scale}\right)}_{i}\left(t-1, \tau\right)
    \varepsilon^{\left(\textrm{scale}\right)}_{i} \left(t\right)\right\rangle
    _{\textrm{average}},
\end{equation}
where the index $i$ corresponds to currency pairs in the market,
$r^{\left(\textrm{scale}\right)}_{i}$ is the return of the pair $i$ in a time
lag $\tau$ in the corresponding scale and
$\varepsilon^{\left(\textrm{scale}\right)}_{i}$ is the trade sign of the pair
$i$ in the corresponding scale. The superscript scale refers to the time scale
used, whether physical time scale ($\textrm{scale} = \textrm{p}$) or trade time
scale ($\textrm{scale} = \textrm{t}$). \textcolor{green}{Finally, the subscript} average refers to
the way to average the price response, whether relative to the physical time
scale ($\textrm{average} = P$) or relative to the trade time scale
($\textrm{average} = T$). The main objective of this work is to analyze the
price response functions for the spot foreign exchange markets.}

For correlated financial markets, \textcolor{red}{the price response function increases to a
maximum and then slowly decreases with increasing $\tau$}. This result is observed empirically in trade
time scale and in physical time scale
\cite{my_paper_response_financial,Wang_2016_avg}.