\section{Foreign exchange market overview}\label{sec:forex_overview}

In Sect. \ref{subsec:key_concepts} we establish the fundamental quantities used
in the price response definitions. In Sect. \ref{subsec:forex_market} we
describe the basic characteristics of the foreign exchange market.

%%%%%%%%%%%%%%%%%%%%%%%%%%%%%%%%%%%%%%%%%%%%%%%%%%%%%%%%%%%%%%%%%%%%%%%%%%%%%%%
\subsection{Key concepts}\label{subsec:key_concepts}

In general, three categories of currency pairs are defined: majors, crosses,
and exotics. The ``major'' foreign exchange currency pairs are the most
frequently traded currencies that are paired with the U.S. dollar (see Table
\ref{tab:majors}). The ``crosses'' are those majors pairs paired between them
and that exclude the U.S. dollar. Finally, the ``exotic'' pairs usually consist
of a major currency alongside a thinly traded currency or an emerging market
economy currency. The majors are the most liquid pairs, in contrast with the
exotics, who can be much more volatile.

In foreign exchange markets, orders are execute at the best available buy or
sell price. Orders often fail to result in an immediate transaction, and are
stored in a queue called the limit order book
\cite{forex_market_micro,forex_structure,stat_prop,predictive_pow,intro_market_micro,prop_order_book}.
The order book is visible for all traders and its main purpose is to ensure
that all traders have the same information on what is offered on the market.
For a detailed description of the operation of the markets, we suggest to see
Ref. \cite{my_paper_response_financial}.

At any given time there is a best (lowest) offer to sell with price
$a\left(t\right)$, and a best (highest) bid to buy with price $b\left(t\right)$
\cite{subtle_nature,prop_order_book,account_spread,limit_ord_spread,stat_theory}.
The price gap between them is called the spread
$s\left(t\right) = a\left(t\right)-b\left(t\right)$
\cite{teach_spread,subtle_nature,Bouchaud_2004,large_prices_changes,market_digest,account_spread,stat_theory,em_stylized_facts}.
Spreads are significantly positively related to price and significantly
negatively related to trading volume. Companies with more liquidity tend to
have lower spreads
\cite{components_spread_tokyo,account_spread,effects_spread,components_spread}.
Despite the foreign exchange market is often cited as the world's largest
financial market, this description fail to consider the considerable differences
in trading volume and liquidity across different currency pairs
\cite{forex_microstructure}. These differences can be directly seen in the
spread. Furthermore, the bid-ask spread is directly related with the
transaction costs to the dealer \cite{teach_spread,spread_futures}.

As there is no price information in the data, we consider a basic definition of
the price given by \cite{forex_liquidity,patterns_forex,political_forex}. The
average of the best ask and the best bid is the midpoint price, which is
defined as
\cite{teach_spread,subtle_nature,Bouchaud_2004,my_paper_response_financial,large_prices_changes,prop_order_book,stat_theory,em_stylized_facts}
\begin{equation}
    m \left(t\right) = \frac{a\left(t\right) + b\left(t\right)}{2}.
\end{equation}
Price changes are typically characterized as returns. If one denotes
$S\left( t\right)$ the price of an asset at time $t$, the return
$r^{\left(g\right)}\left(t, \tau\right)$, at time $t$ and time lag $\tau$ is
simply the relative variation of the price from $t$ to $t + \tau$
\cite{subtle_nature,empirical_facts,asynchrony_effects_corr,tick_size_impact,causes_epps_effect,non_stationarity},
\begin{equation}\label{eq:return_general}
    r^{\left(g\right)} \left(t, \tau \right) = \frac{S\left(t + \tau\right)
    - S\left(t\right)}{S\left(t\right)}.
\end{equation}
We define the returns via the midpoint price as
\begin{equation}\label{eq:midpoint_price_return}
    r\left(t,\tau\right) = \frac{m\left(t+\tau\right)-m\left(t\right)}
    {m\left(t\right)}.
\end{equation}
The distribution of returns is strongly non-Gaussian and its shape continuously
depends on the return period $\tau$. Small $\tau$ values have fat tails return
distributions \cite{subtle_nature}. The trade signs are defined for general
cases as
\begin{equation}\label{eq:trade_sign_general}
    \varepsilon\left(t\right)=\text{sign}\left(S\left(t\right)
    -m\left(t-\delta\right)\right),
\end{equation}
where $\delta$ is a positive time increment. Hence we have
\begin{equation}\label{eq:trade_sign_results}
    \varepsilon\left(t\right)=\left\{
    \begin{array}{cc}
    +1, & \text{If } S\left(t\right)
    \text{ is higher than the last } m\left( t \right)\\
    -1, & \text{If } S\left(t\right)
    \text{ is lower than the last } m\left( t \right)
    \end{array}\right. .
\end{equation}
Here, $\varepsilon(t) = +1$ indicates that the trade was triggered by a market
order to buy and a trade triggered by a market order to sell yields
$\varepsilon(t) = -1$
\cite{subtle_nature,Bouchaud_2004,spread_changes_affect,quant_stock_price_response,order_flow_persistent}.

The main objective of this work is to analyze the price response functions. In
general we define the price response functions in a foreign exchange market as
\begin{equation}\label{eq:response_general}
    R^{\left(\textrm{scale}\right)}_{ii}\left(\tau\right)=\left\langle
    r^{\left(\textrm{scale}\right)}_{i}\left(t-1, \tau\right)
    \varepsilon^{\left(\textrm{scale}\right)}_{i} \left(t\right)\right\rangle
    _{\textrm{average}},
\end{equation}
where the index $i$ correspond to currency pairs in the market,
$r^{\left(\textrm{scale}\right)}_{i}$ is the return of the pair $i$ in a time
lag $\tau$ in the corresponding scale and
$\varepsilon^{\left(\textrm{scale}\right)}_{i}$ is the trade sign of the pair
$i$ in the corresponding scale. The superscript scale refers to the time scale
used, whether physical time scale ($\textrm{scale} = \textrm{p}$) or trade time
scale ($\textrm{scale} = \textrm{t}$). Finally, The subscript average refers to
the way to average the price response, whether relative to the physical time
scale ($\textrm{average} = P$) or relative to the trade time scale
($\textrm{average} = T$).

For correlated financial markets, the price response function increase to a
maximum and then slowly decrease. This result is observed empirically in trade
time scale and in physical time scale
\cite{my_paper_response_financial,Wang_2016_avg}.

%%%%%%%%%%%%%%%%%%%%%%%%%%%%%%%%%%%%%%%%%%%%%%%%%%%%%%%%%%%%%%%%%%%%%%%%%%%%%%%
\subsection{Foreign exchange market}\label{subsec:forex_market}

The foreign exchange market is the most volatile, liquid and largest of all
financial markets
\cite{forex_liquidity,info_forex,forex_market_micro,intraday_forex}, and it has
a paramount importance for the world economy. It affects employment, inflation,
international capital flows, among others \cite{forex_structure}. The foreign
exchange market is a decentralized market without a common trading floor
\cite{info_forex,forex_market_micro,forex_structure,teach_spread}

The term pip (Price Increment Point) is commonly used in the foreign exchange
market instead of tick. The precise definition of a pip is a matter of
convention. Usually, it refers to the incremental value in the fifth non-zero
digit position from the left. It is not related to the position of the decimal
point. For example, one pip in the exchange rate USD/JPY of 124.21 would be
0.01, while one pip for EUR/USD of 1.1021 would be 0.0001
\cite{forex_market_micro,forex_structure,order_flow_forex,micro_eff}.