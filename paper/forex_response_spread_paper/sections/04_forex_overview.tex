\section{Foreign exchange market overview}\label{sec:forex_overview}

In Sect. \ref{subsec:key_concepts} we establish the fundamental quantities used
in the price response definitions. In Sect. \ref{subsec:forex_market} we
describe the basic characteristics of the foreign exchange market.

%%%%%%%%%%%%%%%%%%%%%%%%%%%%%%%%%%%%%%%%%%%%%%%%%%%%%%%%%%%%%%%%%%%%%%%%%%%%%%%
\subsection{Key concepts}\label{subsec:key_concepts}

In general, three categories of currency pairs are defined: majors, crosses,
and exotics. The ``major'' foreign exchange currency pairs are the most
frequently traded currencies that are paired with the U.S. dollar (see Table
\ref{tab:majors}). The ``crosses'' are those majors pairs paired between them
and that exclude the U.S. dollar. Finally, the ``exotic'' pairs usually consist
of a major currency alongside a thinly traded currency or an emerging market
economy currency. The majors are the most liquid pairs, in contrast with the
exotics, who can be much more volatile.

In foreign exchange markets, orders are execute at the best available buy or
sell price. Orders often fail to result in an immediate transaction, and are
stored in a queue called the limit order book
\cite{forex_market_micro,forex_structure,stat_prop,predictive_pow,intro_market_micro,prop_order_book}.
The order book is visible for all traders and its main purpose is to ensure
that all traders have the same information on what is offered on the market.
For a detailed description of the operation of the markets, we suggest to see
Ref. \cite{my_paper_response_financial}.

At any given time there is a best (lowest) offer to sell with price
$a\left(t\right)$, and a best (highest) bid to buy with price $b\left(t\right)$
\cite{subtle_nature,prop_order_book,account_spread,limit_ord_spread,stat_theory}.
The price gap between them is called the spread
$s\left(t\right) = a\left(t\right)-b\left(t\right)$
\cite{teach_spread,subtle_nature,Bouchaud_2004,large_prices_changes,market_digest,account_spread,stat_theory,em_stylized_facts}.
Spreads are significantly positively related to price and significantly
negatively related to trading volume. Companies with more liquidity tend to
have lower spreads
\cite{components_spread_tokyo,account_spread,effects_spread,components_spread}.
Despite the foreign exchange market is often cited as the world's largest
financial market, this description fail to consider the considerable differences
in trading volume and liquidity across different currency pairs
\cite{forex_microstructure}. These differences can be directly seen in the
spread. Furthermore, the bid-ask spread is directly related with the
transaction costs to the dealer \cite{teach_spread,spread_futures}.

As there is no price information in the data, we consider a basic definition of
the price given by \cite{forex_liquidity,patterns_forex,political_forex}. The
average of the best ask and the best bid is the midpoint price, which is
defined as
\cite{teach_spread,subtle_nature,Bouchaud_2004,my_paper_response_financial,large_prices_changes,prop_order_book,stat_theory,em_stylized_facts}
\begin{equation}
    m \left(t\right) = \frac{a\left(t\right) + b\left(t\right)}{2}.
\end{equation}
Price changes are typically characterized as returns. If one denotes
$S\left( t\right)$ the price of an asset at time $t$, the return
$r^{\left(g\right)}\left(t, \tau\right)$, at time $t$ and time lag $\tau$ is
simply the relative variation of the price from $t$ to $t + \tau$
\cite{subtle_nature,empirical_facts,asynchrony_effects_corr,tick_size_impact,causes_epps_effect,non_stationarity},
\begin{equation}\label{eq:return_general}
    r^{\left(g\right)} \left(t, \tau \right) = \frac{S\left(t + \tau\right)
    - S\left(t\right)}{S\left(t\right)}.
\end{equation}
We define the returns via the midpoint price as
\begin{equation}\label{eq:midpoint_price_return}
    r\left(t,\tau\right) = \frac{m\left(t+\tau\right)-m\left(t\right)}
    {m\left(t\right)}.
\end{equation}
The distribution of returns is strongly non-Gaussian and its shape continuously
depends on the return period $\tau$. Small $\tau$ values have fat tails return
distributions \cite{subtle_nature}. The trade signs are defined for general
cases as
\begin{equation}\label{eq:trade_sign_general}
    \varepsilon\left(t\right)=\text{sign}\left(S\left(t\right)
    -m\left(t-\delta\right)\right),
\end{equation}
where $\delta$ is a positive time increment. Hence we have
\begin{equation}\label{eq:trade_sign_results}
    \varepsilon\left(t\right)=\left\{
    \begin{array}{cc}
    +1, & \text{If } S\left(t\right)
    \text{ is higher than the last } m\left( t \right)\\
    -1, & \text{If } S\left(t\right)
    \text{ is lower than the last } m\left( t \right)
    \end{array}\right. .
\end{equation}
Here, $\varepsilon(t) = +1$ indicates that the trade was triggered by a market
order to buy and a trade triggered by a market order to sell yields
$\varepsilon(t) = -1$
\cite{subtle_nature,Bouchaud_2004,spread_changes_affect,quant_stock_price_response,order_flow_persistent}.

The main objective of this work is to analyze the price response functions. In
general we define the price response functions in a foreign exchange market as
\begin{equation}\label{eq:response_general}
    R^{\left(\textrm{scale}\right)}_{ii}\left(\tau\right)=\left\langle
    r^{\left(\textrm{scale}\right)}_{i}\left(t-1, \tau\right)
    \varepsilon^{\left(\textrm{scale}\right)}_{i} \left(t\right)\right\rangle
    _{\textrm{average}},
\end{equation}
where the index $i$ correspond to currency pairs in the market,
$r^{\left(\textrm{scale}\right)}_{i}$ is the return of the pair $i$ in a time
lag $\tau$ in the corresponding scale and
$\varepsilon^{\left(\textrm{scale}\right)}_{i}$ is the trade sign of the pair
$i$ in the corresponding scale. The superscript scale refers to the time scale
used, whether physical time scale ($\textrm{scale} = \textrm{p}$) or trade time
scale ($\textrm{scale} = \textrm{t}$). Finally, The subscript average refers to
the way to average the price response, whether relative to the physical time
scale ($\textrm{average} = P$) or relative to the trade time scale
($\textrm{average} = T$).

For correlated financial markets, the price response function increase to a
maximum and then slowly decrease. This result is observed empirically in trade
time scale and in physical time scale
\cite{my_paper_response_financial,Wang_2016_avg}.

%%%%%%%%%%%%%%%%%%%%%%%%%%%%%%%%%%%%%%%%%%%%%%%%%%%%%%%%%%%%%%%%%%%%%%%%%%%%%%%
\subsection{Foreign exchange market}\label{subsec:forex_market}

The foreign exchange market is a 24-hour global decentralized or
over-the-counter (OTC) market for the trading of currencies closing only on the
weekends.
The foreign exchange market is the most volatile, liquid and largest of all
financial markets
\cite{forex_liquidity,info_forex,forex_market_micro,forex_structure,teach_spread,intraday_forex,book_forex,book_forex_2,book_forex_3},
and it has a paramount importance for the world economy. It affects employment,
inflation, international capital flows, among others \cite{forex_structure}.
The major participants trading in this market include governments, central
banks, global funds, retail clients and corporations
\cite{book_forex_2,book_forex_3}. Trading of currency in the foreign exchange
market involves the purchase and sale of two currencies at the same time
\cite{book_forex,book_forex_2,book_forex_3}. The price one currency can be
exchanged with another currency is the foreign exchange rate.

In the foreign exchange market, the trading day begins in Australia and Asia.
Then the markets in Europe open and finally the markets in America
\cite{forex_market_micro,forex_structure,book_forex_2,book_forex_3}. As the
market close time in New York overlaps the market open time in Australia and
Asia, the markets do not formally close during the week. Thus, using the New
York time as reference, the market opens on Sunday at 19h00 and closes on
Friday at 17h00.

Currency markets are divided into spot market, forward, future, currency swaps
and currency options \cite{book_forex_2,book_forex_3,book_forex_4}. In our work
we particularly focus on the spot market, where as his name suggest, the trades
are settled on the spot \cite{book_forex,book_forex_3}. The transactions in
this market represent up to the 40\% of the total market transactions in the
foreign exchange market \cite{book_forex}. The most traded currencies in the
spot market are the US dollar, euro, Japanese yen, British pound and Swiss
franc.

The term pip (Price Increment Point) is commonly used in the foreign exchange
market instead of tick. The precise definition of a pip is a matter of
convention. Usually, it refers to the incremental value in the fifth non-zero
digit position from the left. It is not related to the position of the decimal
point. For example, one pip in the exchange rate USD/JPY of 124.21 would be
0.01, while one pip for EUR/USD of 1.1021 would be 0.0001
\cite{forex_market_micro,forex_structure,order_flow_forex,micro_eff,book_forex_3}.

***
London, New York, and Tokyo dominate foreign exchange trading \cite{book_forex_4}. The currency markets are the largest and most liquid of all the financial markets

The main participants in this market are the larger international banks. Financial centers around the world function as anchors of trading between a wide range of multiple types of buyers and sellers around the clock, with the exception of weekends.

To allow the buying and selling of currencies, the foreign exchange market has a network of different currency  traders  who  work  around  the  clock  to  complete  the  forex  transactions.  The  main  types  of  foreign exchange markets are spot market, forward, future, currency swaps, and currency options \cite{book_forex_2,book_forex_3}. The spot and forward exchange markets are OTC markets

In a spot market, a specified amount of cur-rency is transferred against the receipt of a specified amount of another currency based on an agreed exchange rate (spot rate) within two business days of the date in which the deal was finalized. The most widely traded currency in volume in the spot market is the US dollar. The other major currencies are the euro, Japanese yen, British pound, and Swiss franc. In a spot market, information regarding open interest and volume is unavailable because the currency transactions are carried in the OTC markets and not through exchanges. Most currencies in the interbank spot market are quoted in European terms in  which  the  US  dollar  is  priced  in  terms  of  the  foreign  currency.  In  an  interbank  foreign  exchange  market, traders buy currency for inventory at bid price and sell from inventory at the higher offer or ask price. The difference between the bid and ask price is called a bid-ask spread \cite{book_forex}.
Spot rate: The current exchange rate. It is the rate that is currently beingcharged. A transaction in the sport market requires an immediate cashsettlement (e.g., either one or two days depending upon the currency). It isalso referred to as the cash market \cite{book_forex_3}.

The commercial banks and investment banks together form the interbank market. The interbank market is the largest market that operates in the foreign exchange market.  Essentially  the  foreign  exchange  market  is  divided  into  whole  (interbank)  and  retail  (client)  markets. The interbank market consists of a network of corresponding banking relationships in which large commercial banks maintain corresponding banking accounts with one another by maintaining de-mand deposit accounts. The Society for Worldwide Interbank Financial Telecommunication (SWIFT) HELPS international commercial banks communicate these transactions. The Clearing House Interbank Payments System (CHIPS) in association with the US Federal Reserve Bank System (Fedwire) pro-vides a clearinghouse for the interbank settlement between international banks. Forex brokers match dealer orders to buy and sell currencies for fees \cite{book_forex}.

Currencies are always traded in pairs, so the "value" of one of the currencies in that pair is relative to the value of the other.
The value of a country's currency depends on whether it is a "free float" or "fixed float." Free-floating currencies are those whose relative value is determined by free-market forces, such as supply-demand relationships. A fixed float is where a country's governing body sets its currency's relative value to other currencies, often by pegging it to some standard. Free-floating currencies include the U.S. dollar, Japanese yen, and British pound, while examples of fixed floating currencies include the Chinese Yuan and the Indian Rupee.
The FX market is a closed system. As one value increases another valuehas to decrease. Unlike stock prices, all FX rates cannot appreciate \cite{book_forex_3}.

The foreign exchange market is unique because of the following characteristics:

    its huge trading volume, representing the largest asset class in the world leading to high liquidity;
    its geographical dispersion;
    its continuous operation: 24 hours a day except for weekends, i.e., trading from 22:00 GMT on Sunday (Sydney) until 22:00 GMT Friday (New York);
    the variety of factors that affect exchange rates;
    the low margins of relative profit compared with other markets of fixed income; and
    the use of leverage to enhance profit and loss margins and with respect to account size.

As such, it has been referred to as the market closest to the ideal of perfect competition, notwithstanding currency intervention by central banks.

The foreign exchange market is the most liquid financial market in the world. Traders include governments and central banks, commercial banks, other institutional investors and financial institutions, currency speculators, other commercial corporations, and individuals. According to the 2019 Triennial Central Bank Survey, coordinated by the Bank for International Settlements, average daily turnover was \$6.6 trillion in April 2019 (compared to \$1.9 trillion in 2004).[3] Of this \$6.6 trillion, \$2 trillion was spot transactions and \$4.6 trillion was traded in outright forwards, swaps, and other derivatives.
Foreign exchange is traded in an over-the-counter market where brokers/dealers negotiate directly with one another, so there is no central exchange or clearing house.
Foreign exchange markets are made up of banks, forex dealers, commercial companies, central banks, investment management firms, hedge funds, retail forex dealers, and investors.

Benefits of Using the Forex Market

There are some key factors that differentiate the forex market from others, like the stock market.

    There are fewer rules, which means investors aren't held to the strict standards or regulations found in other markets.
    There are no clearing houses and no central bodies that oversee the forex market.
    Most investors won't have to pay the traditional fees or commissions that you would on another market.
    Because the market is open 24 hours a day, you can trade at any time of day, which means there's no cut-off time to be able to participate in the market.
    Finally, if you're worried about risk and reward, you can get in and out whenever you want, and you can buy as much currency as you can afford based on your account balance and your broker's rules for leverage.

Unlike a stock market, the foreign exchange market is divided into levels of access. At the top is the interbank foreign exchange market, which is made up of the largest commercial banks and securities dealers. Within the interbank market, spreads, which are the difference between the bid and ask prices, are razor sharp and not known to players outside the inner circle. The difference between the bid and ask prices widens (for example from 0 to 1 pip to 1–2 pips for currencies such as the EUR) as you go down the levels of access. This is due to volume. If a trader can guarantee large numbers of transactions for large amounts, they can demand a smaller difference between the bid and ask price, which is referred to as a better spread. The levels of access that make up the foreign exchange market are determined by the size of the "line" (the amount of money with which they are trading). The top-tier interbank market accounts for 51\% of all transactions.[61] From there, smaller banks, followed by large multi-national corporations (which need to hedge risk and pay employees in different countries), large hedge funds, and even some of the retail market makers.
One of the biggest differences between the FX marketsand other financial markets is the overall activity from corporations to facilitate day-to-day business practices as well as to hedge longer-term risk. Corporations willengage in FX trading to facilitate necessary business transactions, to hedge againstmarket risk, and, to a lesser extent, to facilitate longer-term investment needs \cite{book_forex_3}.

In general there are many flexible exchange rate systems. In a free-floating or independent-floating currency,  the  exchange  rate  is  determined  by  the  market,  with  foreign  exchange  intervention  occur-ring only to prevent undue fluctuations. In a managed-floating system, the central monetary author-ity of countries influences the movement of the exchange rate through active intervention in the forex market with no preannounced path for the exchange rate n a fixed-peg arrangement, the country pegs its currency at a fixed rate to a major currency or to a basket of currencies \cite{book_forex}

Exchange rates are basically determined by the demand and supply of a particular currency as com-pared to other currencies. Inflation is a major factor that affects exchange rate. If a country has low inflation, its domestic currency will appreciate in value as the purchasing power of the currency increases as  compared  to  other  currencies.  Inflation  and  interest  rates  are  highly  correlated.  Higher  inflation  will lead to higher interest rates in an economy. Interest rate is also a critical determinant of changes in exchange rates. A current account deficit or negative BOP indicates that there is excess demand for foreign currency, which would lead to lower value of a country’s currency \cite{book_forex}

The existingspread in any currency will vary according to the individual currencytrader, the currency being traded, and the trading bank’s overall view ofconditions in the foreign exchange market. The spread quoted will tendto increase for more thinly traded currencies (i.e., currencies that do notgenerate a large volume of trading) or when the bank perceives that therisks associated with trading in a currency at a particular time are rising \cite{book_forex_2}.

In the wholesale banking foreign exchange market,the bid-offer spread is the only transaction cost \cite{book_forex_2}

***