\section{Price response functions}\label{sec:response_functions}

In Sect. \ref{subsec:response_function_trade} we analyze the responses
functions in trade time scale and in Sect.
\ref{subsec:response_function_physical} we analyze the responses functions in
physical time scale.


%%%%%%%%%%%%%%%%%%%%%%%%%%%%%%%%%%%%%%%%%%%%%%%%%%%%%%%%%%%%%%%%%%%%%%%%%%%%%%%
\subsection{Response functions on trade time scale}
\label{subsec:response_function_trade}

The price response function in trade time scale is defined as
\cite{my_paper_response_financial}
\begin{equation}\label{eq:response_functions_trade_scale_general}
    R^{\left(\textrm{t}\right)}_{i}\left(\tau\right)=\left\langle
    r^{\left(\textrm{t}\right)}_{i}\left(t-1,\tau \right)
    \varepsilon_{i}^{\left(\textrm{t}\right)}
    \left(t, n\right)\right\rangle _{T}.
\end{equation}
To compute the response functions on trade time scale, we use both, the trade
signs and the returns from the tick-by-tick original data during a week in
market time. Then, the response is averaged by the number of trades.

\begin{figure}[htbp]
    \centering
    \includegraphics[width=\columnwidth]
    {figures/04_responses_trade_scale.png}
    \caption{Price response functions
             $R^{\left(\textrm{t}\right)}_{i}\left(\tau\right)$ versus time
             lag $\tau$ on a logarithmic scale in trade time scale for the
             years 2008 (top), 2014 (middle) and 2019 (bottom).}
    \label{fig:response_function_trade_scale}
\end{figure}

The results of Fig. \ref{fig:response_function_trade_scale} show the
price response functions of the seven foreign exchange major pairs used in the
analysis (see Table \ref{tab:majors}) for three different years. The results
found for all the years are entirely in line with price responses seen in other
financial markets, particularly with correlated financial markets. The response
functions have an initial increasing trend to a maximum, that flattens out and
saturates at some level, and eventually slowly decrease. This shape is
explained by an initial increase caused by autocorrelated transaction flow. The
flattening out is due to the market liquidity adapting to this flow and assuring
diffusive prices \cite{EMH_lillo}. For our selected pairs, a time lag of
$\tau = 10^{3} $ trades is enough to see an increase to a maximum followed by a
decrease. Thus, the trend in the price response functions is eventually
reversed. The response signal is much more noisier in the year 2008 for the
first seconds in the time lag. This behavior is because of the smaller amount
of data of the corresponding year. In general, more data was recorded in recent
years than in past years. In the three years analyzed, the more liquid currency
pairs have a smaller response in comparison with the non-liquid pairs.  The
strength of the response function varies from one year to the other. In 2008
the strength of the signal was one order of magnitude stronger than the
response in 2014, but the signals in 2014 have approximately twice the strength
of the signals of 2019. This behavior can be explained by the fact that in
recent times algorithm trading has been used intensively. Thus, many more
trades were carried out in the last years, which means, the impact of each
trade is reduced, and then the response functions tend to decrease compared
with previous years.

%%%%%%%%%%%%%%%%%%%%%%%%%%%%%%%%%%%%%%%%%%%%%%%%%%%%%%%%%%%%%%%%%%%%%%%%%%%%%%%
\subsection{Response functions on physical time scale}
\label{subsec:response_function_physical}

One important detail to compute the price response function on physical time
scale is to define how the averaging of the function will be made, because the
response functions highly differ when we include or exclude
$\varepsilon^{\left(\textrm{p}\right)}_j \left( t\right) = 0$
\cite{Wang_2016_cross}. The price responses including
$\varepsilon^{\left(\textrm{p}\right)}_j \left( t\right) = 0$ are weaker than
the excluding ones due to the omission of direct influence of the lack of
trades. However, either including or excluding
$\varepsilon^{\left(\textrm{p}\right)}_j \left( t\right) = 0$ does not change
the trend of price reversion versus the time lag, but it does affect the
response function strength \cite{Wang_2016_avg}. For a deeper analysis of the
influence of the term
$\varepsilon^{\left(\textrm{p}\right)}_j \left( t\right) = 0$ in price response
functions, we suggest reviewing Refs. \cite{Wang_2016_avg,Wang_2016_cross}. We
will only take into account the price response functions excluding
$\varepsilon^{\textrm{p}}_j \left( t\right) = 0$.

We define the price response functions on physical time scale, using
the trade signs and the returns sampled in seconds from the original data on
physical time scale. The price response function on physical time scale is
defined as \cite{my_paper_response_financial}
\begin{equation}\label{eq:response_functions_time_scale_general}
    R^{\left(\textrm{p}\right)}_{i}\left(\tau\right)=\left\langle
    r^{\left(\textrm{p}\right)}_{i}\left(t-1, \tau\right)
    \varepsilon_{i}^{\left(\textrm{p}\right)} \left(t\right)\right\rangle _{P}
\end{equation}
\begin{figure}[htbp]
    \centering
    \includegraphics[width=\columnwidth]
    {figures/04_responses_physical_scale.png}
    \caption{Price response functions
             $R^{\left(\textrm{p}\right)}_{i}\left(\tau\right)$ excluding
             $\varepsilon^{\left(\textrm{p}\right)}_{i}\left(t\right) = 0$ versus time
             lag $\tau$ on a logarithmic scale in physical time scale for the
             years 2008 (top), 2014 (middle) and 2019 (bottom).}
    \label{fig:response_function_physical_scale}
\end{figure}
The results shown in Fig. \ref{fig:response_function_physical_scale} are the
price response functions on physical time scale for three different years. The
results show approximately the same behavior observed in currency exchange
pairs in trade time scale, and in correlated financial markets, where we can
see that an increase to a maximum is followed by a decrease. Thus again, the
trend in the price responses is eventually reversed. An exception occurs in the
year 2008, where the response at short time lags seems to decrease, to then
start to slightly increase, and finally it decreases again.

The price response functions on physical time scale are smoother than the
responses on trade time scale. As we reduce from trade data all the returns and
trade signs in one second to one data point on physical time scale, and as this
sampling gives the same weight to every data point, the curves look smoother.

Compared with the response functions on trade time scale, the strength of the
signal of the response functions on physical time scale are similar in
magnitude in the corresponding years. Thus, the strength of the signal in 2008
for trade time scale is similar to the strength of the signal in 2008 for
physical time scale, and so on. \textcolor{red}{This behavior is different from the one presented
in correlated financial markets, where the results differ about a factor of
two depending on the time scale \cite{my_paper_response_financial}.}

On physical time scale, \textcolor{red}{we can see that the liquid pairs have a smaller price
response compared with non-liquid pairs. The liquidity of the pairs vary regarding
the analyzed year. For the years 2008 and 2014, the most liquid pairs are the EUR/USD
and the GBP/USD. For 2019 the most liquid pairs are EUR/USD and USD/CAD.} Therefore, the price response of a
foreign exchange pair with large activity is smaller to the small impact of
each trade. \textcolor{red}{Also, the former year responses have stronger signals.} We consider
the same argument of algorithm trading to explain why the signals in recent
years are weaker than in older years.
