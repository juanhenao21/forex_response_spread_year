\section{Conclusion}\label{sec:conclusion}

Price response functions provide quantitative information on the deviation from
Markovian behavior. They measure price changes resulting from execution of
market orders. We used these functions in big data analysis for spot foreign
exchange markets. Such a study was, to the best of our knowledge, never done
before.

We analyzed price response functions in spot foreign exchange markets for
different years and different time scales. We used trade time scale and
physical time scale to compute the price response functions for the seven major
foreign exchange pairs for three different years. These major pairs are highly
relevant in the dynamics of the market. The use of different time scales and
calendar years in the work had the intention to display the different behaviors
the price response function could take when the time parameters differ.

The price response functions were analyzed according to the time scales. On
trade time scale, the signals were noisier. For both time scales we observe
that the signal for all the pairs increases to a maximum and then starts to
slowly decrease. However, for the year 2008 the shape of the signals is not as
well defined as in the other years. The increase-decrease behavior observed in
the spot foreign exchange market was also reported in correlated financial
markets \cite{my_paper_response_financial,Wang_2016_avg}. These results show
that the price response functions conserve their behavior in different years
and in different markets. Price response function shape is qualitatively
explained considering an initial increase caused by the autocorrelated
transaction flow. To assure diffusive prices, price response flattens due to
market liquidity adapting to the flow in the initial increase.

On both scales, the more liquid pairs have a smaller price response function
compared with the non-liquid pairs. As the liquid pairs have more trades during
the market time, the impact of each trade is reduced. Comparing years and
scales, the price response signal is stronger in past than in recent years. As
algorithmic trading has gained great relevance, the quantity of trades has
grown in recent years, and in consequence, the impact in the response has
decreased.

Finally, we checked the pip spread impact in price response functions for three
different years. We used 46 foreign exchange pairs and grouped them depending
on the conditions of the corresponding year analyzed. We employ the year
average pip spread of every pair for each year. For all the year and time
scales, the price response function signals were stronger for the groups of
pairs with larger pip spreads and weaker for the group of pairs with smaller
spreads. For the average of the price response functions, it was only possible
to see the increase-maximum-decrease behavior in the year 2015 in both scales,
and in the year 2019 on trade time scale. Hence, the noise in the cross and
exotic pairs due to the lack of trading compared with the majors seems
stronger. A general average price response behavior for each year and time
scale was spotted for the groups, suggesting a market effect on the foreign
exchange pairs in each year.

Comparing the response functions in stock and spot currency exchange markets
from a more general viewpoint, we find a remarkable similarity. It triggers the
conclusion that the order book mechanism generates in a rather robust fashion
the observed universal features in these two similar, yet different subsystems
within the financial system.