\section{Introduction}\label{sec:introduction}

Complex systems composed of many interacting subsystems are of high interest
among the scientific community. Furthermore, economics systems are complex
interacting systems for which a tremendous amount of quantitative data exists
\cite{physicists_contribution}. In recent years, econophysicists have been
applying to economic phenomena various models and concepts associated with
physics of complex systems. Particularly, they use statistical physics to
analyze systems made of many particles (investors, traders, consumers,
and so on). Thus, markets are viewed as macroscopic complex systems with an
internal microscopic structure consisting of many of these agents interacting
\cite{complex_markets}. Among the complex systems in the economic phenomena,
we center our attention in the foreign exchange market. The emphasis of the
present interdisciplinary contribution is to study the application of the price
response function concept and shed light on the impact of the trades in the
market. Such analysis was already made for stock markets, but for our surprise,
we did not find any similar analysis on foreign exchange markets. Then, we feel
this study is a rewarding effort. It helps to examine the characteristics of
the functions applied to the foreign exchange market and it is suitable to
compare the similarities and differences to other markets.

The foreign exchange market is the most volatile, liquid and largest of all
financial markets
\cite{forex_liquidity,info_forex,forex_market_micro,intraday_forex}, and it has
a paramount importance for the world economy. It affects employment, inflation,
international capital flows, among others \cite{forex_structure}. The foreign
exchange market is a decentralized market without a common trading floor
\cite{teach_spread,forex_structure,info_forex,forex_market_micro}

The term pip (Price Increment Point) is commonly used in the foreign exchange
market instead of tick. The precise definition of a pip is a matter of
convention. Usually, it refers to the incremental value in the fifth non-zero
digit position from the left. It is not related to the position of the decimal
point. For example, one pip in the exchange rate USD/JPY of 124.21 would be
0.01, while one pip for EUR/USD of 1.1021 would be 0.0001
\cite{order_flow_forex,forex_structure,micro_eff,forex_market_micro}.

The foreign exchange market has attracted a lot of attention in the last 20
years. Electronic trading has changed an opaque market to a fairly transparent
one with transactions costs that are a fraction of their former level. The
large amount of data that is now available to the public make possible
different kinds of data analysis. Intense research is currently carried out in
different directions
\cite{curr_speculation,forex_algorithmic,teach_spread,electronic_forex,forex_microstructure,patterns_forex,eur_change_forex,spread_competition,forex_structure,political_forex,forex_liquidity,forex_volatility,info_forex,local_forex,intraday_forex,forex_inefficiency}.

McGroarty et al. \cite{micro_eff} found that smaller volumes cause larger
bid-ask spreads for technical reasons related to the measurement, whereas Hau
et al. \cite{eur_int_curr,eur_change_forex} claim that larger bid-ask spreads
caused smaller volumes due to the traders' behavior.

Burnside et al. \cite{curr_speculation} found the spreads to be between two and
four times larger for emerging market currencies than for developed country
currencies. According to Huang and Masulis \cite{spread_competition}, bid-ask
spreads increase when the foreign exchange market volatility increases, and
decrease when the competition between the dealers increases. Ding and Hiltrop
\cite{electronic_forex} showed that the Electronic Broking Services (EBS)
reduces spreads significantly, but dealers with information advantage tend to
quote relatively wider spreads. King \cite{spread_futures} analyzed the foreign
exchange futures market and observed that the number of transactions is
negatively related with bid-ask spread, whereas volatility in general is
positively related. Serbinenko and Rachev \cite{intraday_forex} focus on the
three major market characteristics, namely efficiency, liquidity and
volatility, and found that the market is efficient in a weak form. Menkhoff and
Schmeling \cite{local_forex} used orders from the Russian interbank for Russian
rouble/US dollar rate. They analyzed the price impact in different regions of
Russia, and found that regions that are centers of political and financial
decision making have high permanent price impact.

The price response functions measures price changes resulting from execution of
market orders. The price response function measures how a buy or sell order at
time $t$ influences on average the price at a later time $t + \tau$. It was
shown in different works
\cite{my_paper_response_financial_arxiv,r_walks_liquidity,subtle_nature,Bouchaud_2004,theory_market_impact,components_spread_tokyo,dissecting_cross,Wang_2016_cross,Wang_2016_avg}


Little is known about price response functions in the foreign exchange markets:
\cite{forex_liquidity,forex_volatility,response_funct_fx}. Melvin and Melvin
\cite{forex_volatility} simulate their proposed model for different foreign
exchange markets region to analyze the impact of a one-standard-deviation shock
using impulse response functions. The general pattern of response was a fairly
steep drop over the first couple of days followed by a few days of gradual
decline until the response is not statistically different from zero. Mancini et
al. \cite{forex_liquidity} model the price impact and return reversal to
analyze liquidity. Their model predicts that more liquid assets should exhibit
narrower spreads and lower price impact.

To the best of our knowledge, no large-scale data analysis of response
functions for foreign exchange market has been carried out. Response functions
are important observables as they give information on non-Markovian behavior.
It is the purpose of the present study to close this gap. Based on a series of
detailed empirical results obtained on trade by trade data, we will show that
the price response functions in the foreign exchange markets behave
qualitatively similar as the ones in correlated stocks markets. We consider
different time scales and currency pairs to compute the price response
functions. We also shed light on the spread impact in the response functions
for currency pairs. To facilitate the reproduction of our results, the source
code for the data analysis is available in Ref. \cite{code}.

The paper is organized as follows: in Sect. \ref{sec:data_set} we present our
data set of foreign exchange pairs and briefly describe the physical and trade
time scale. We define the time scale to be used in Sect. \ref{sec:time_scale},
and compute the price response functions for the majors pairs in Sect.
\ref{sec:response_functions}. In Sect. \ref{sec:spread_impact} we show how
the spread impact the values of the response functions. Our conclusions follow
in Sect. \ref{sec:conclusion}.