\section{Introduction}\label{sec:introduction}

A major objective of data driven research on complex systems is the
identification of generic or universal statistical behavior. The tremendous
success of thermodynamics and statistical mechanics serves as an inspiration
when continuing this quest in complex systems beyond traditional physics.
Particularly interesting are large complex systems which consist of similar,
yet clearly distinguishable complex subsystems. Financial markets, for example,
have well defined subsystems as foreign exchange markets, stock markets, bond
markets, among others. The degree of universality found in one particular
subsystem can then be assessed if this type of universality is also seen in
another subsystem. If applicable, useful information on the impact of specific
system features on this universality may then be inferred.

In spite of the considerable interest, a thorough statistical analysis of
the microstructure in foreign exchange markets was hampered by limited access
to data. This changed, and nowadays such data analyses are possible down to the
level of ticks and over long time scales.

\textcolor{red}{Here, we carry out such a study for finance, because  a tremendous
amount of data is available in general
\cite{physicists_contribution,data_01,data_02,data_03,data_04}, and in particular
in foreign exchange markets \cite{data_05,forex_microstructure,data_06}, to cite some
examples.} Markets may be viewed as
macroscopic complex systems with an internal microscopic structure that is to a
large extent accessible by big data analysis \cite{complex_markets}.
Stock markets and foreign exchange markets are clearly distinct, but share many
common features. In previous analyses, we studied response functions in stock
markets to shed light on non-Markovian behavior. Here, we extend that to the
spot foreign exchange markets. To our surprise, we did not find such an
investigation in the literature. Hence, we believe that this study is a
rewarding effort. It helps to examine the behavior of the functions applied to
the foreign exchange market and it is suitable to compare the similarities and
differences to other markets.

The foreign exchange market has attracted a lot of attention in the last 20
years. Electronic trading has changed an opaque market to a fairly transparent
one with transaction costs that are a fraction of their former level. The large
amount of data that is now available to the public makes possible different
kinds of data analysis. Intense research is currently carried out in different
directions
\cite{forex_liquidity,info_forex,intraday_forex,forex_structure,teach_spread,forex_microstructure,electronic_forex,forex_algorithmic,curr_speculation,patterns_forex,eur_change_forex,spread_competition,political_forex,forex_volatility,local_forex,forex_inefficiency}.

McGroarty et al. \cite{micro_eff} found that smaller volumes cause larger
bid-ask spreads for technical reasons related to the measurement, whereas Hau
et al. \cite{eur_change_forex,eur_int_curr} claim that larger bid-ask spreads
caused smaller volumes due to the traders' behavior.

Burnside et al. \cite{curr_speculation} found the bid-ask spreads to be between
two and four times larger for emerging market currencies than for developed
country currencies. According to Huang and Masulis \cite{spread_competition},
bid-ask spreads increase when the foreign exchange market volatility increases,
and decrease when the competition between the dealers increases. Ding and
Hiltrop \cite{electronic_forex} showed that the Electronic Broking Services
(EBS) reduces bid-ask spreads significantly, but dealers with information
advantage tend to quote relatively wider bid-ask spreads. King
\cite{spread_futures} analyzed the foreign exchange futures market and observed
that the number of transactions is negatively related with bid-ask spread,
whereas volatility in general is positively related. Serbinenko and Rachev
\cite{intraday_forex} focus on the three major market characteristics, namely
efficiency, liquidity and volatility, and found that the market is efficient in
a weak form. Menkhoff and Schmeling \cite{local_forex} used orders from the
Russian interbank for Russian rouble/US dollar rate. They analyzed the price
impact in different regions of Russia, and found that regions that are centers
of political and financial decision making have high permanent price impact.

Price response functions are a powerful tool to obtain dynamical information
because they measure price changes implied by execution of market orders.
Specifically, they measure how a buy or sell order at time $t$ influences on
average the price at a later time $t + \tau$. It was shown in different works
\cite{components_spread_tokyo,dissecting_cross,r_walks_liquidity,subtle_nature,Bouchaud_2004,theory_market_impact,my_paper_response_financial,Wang_2016_avg,Wang_2016_cross}
that the price response functions increase to a maximum and then slowly
decrease as the time lag grows.

Little is known about price response functions or related quantities in the
foreign exchange markets
\cite{forex_liquidity,forex_volatility,response_funct_fx}. Melvin and Melvin
\cite{forex_volatility} simulate their proposed model for different foreign
exchange markets region to analyze the impact of a one-standard-deviation shock
using impulse response functions. The general pattern of response was a fairly
steep drop over the first couple of days followed by a few days of gradual
decline until the response is not statistically different from zero. Mancini et
al. \cite{forex_liquidity} model the price impact and return reversal to
analyze liquidity. Their model predicts that more liquid assets should exhibit
narrower bid-ask spreads and lower price impact.

To the best of our knowledge, no large-scale data analysis of response
functions for the spot foreign exchange market has been carried out. Response
functions are important observables as they give information on non-Markovian
behavior. It is the purpose of the present study to close this gap. Based on a
series of detailed empirical results obtained on trade by trade data, we show
that the price response functions in the foreign exchange markets behave
qualitatively similar as the ones in correlated stocks markets. We consider
different time scales, years and currency pairs to compute the price response
functions. Finally, we shed light on the bid-ask spread impact in the response
functions for foreign exchange pairs. We use a pip bid-ask spread definition to
group different foreign exchange pairs and show that large pip bid-ask spreads
have a stronger impact on the response. To facilitate the reproduction of our
results, the source code for the data analysis is available in Ref.
\cite{code}.

The paper is organized as follows: in Sect. \ref{sec:forex_overview} we
introduce the foreign exchange market. In Sect. \ref{sec:data_set} we present
our data set of spot foreign exchange pairs and briefly describe the physical
and trade time scale. We define the time scale to be used in Sect.
\ref{sec:time_scale}, and compute the price response functions for foreign
exchange pairs in Sect.  \ref{sec:response_functions}. In Sect.
\ref{sec:spread_impact} we show how the bid-ask spread impact the values of the
response functions. Our conclusions follow in Sect. \ref{sec:conclusion}.