\section{Data set}\label{sec:data_set}



The foreign exchange financial data was obtained from
\href{www.histdata.com}{HistData.com}. We use a tick-by-tick database in
generic ASCII format for different years and currency pairs. The data comprises
the date time stamp (YYYYMMDD HHMMSSNNN), the best bid and best ask quotes
prices in the Eastern Standard Time (EST) time zone. No information about the
size of each transaction is provided. Also, the identity of the participants is
not given.

***
Trading Volumes, in forex, are not aggregated and the only volume that you can
find is the Broker Specific Volumes. So, therefore, we decided to remove the
volume information from the delivered data.

The Ask price is only included in the Tick data of the Generic ASCII format.
With both Bid and Ask you’ll have the spread value for each specific tick.

For tick data files, here are the fields order of the supplied data:

DateTime,Bid,Ask,Volume
***

On the physical time scale, for each exchange rate, we process the irregularly
spaced raw data to construct second-by-second price and volume series, each
containing 86,400 observations per day. For every second, the midpoint of best
bid and ask quotes or the transaction price of deals is used to construct
one-second log-returns.

To analyze the price response functions in Sect. \ref{sec:response_functions},
we select the seven major currency pairs in three different years (2008, 2014
and 2019). Table \ref{tab:majors} shows the currency pairs with their
corresponding symbols.

\begin{table}[htbp]
\centering
\begin{threeparttable}
\caption{Analyzed currency pairs.}
\begin{tabular*}{\columnwidth}{P{5cm}P{3cm}}
\toprule
\bf{Currency pair} & \bf{Symbol} \tabularnewline
\midrule
euro/U.S dollar& EUR/USD \tabularnewline
British pound/U.S dollar& GBP/USD \tabularnewline
Japanese yen/U.S dollar& JPY/USD \tabularnewline
Australian dollar/U.S dollar& AUD/USD \tabularnewline
U.S dollar/Swiss franc& USD/CHF \tabularnewline
U.S dollar/Canadian dollar& USD/CAD \tabularnewline
New Zealand dollar/U.S dollar& NZD/USD \tabularnewline
\bottomrule
\end{tabular*}
\label{tab:majors}
\end{threeparttable}
\end{table}

To analyze the spread impact in price response functions (Sect.
\ref{sec:spread_impact}), we select 46 currency pairs, in three different years
(2011, 2015 and 2019). The selected pairs are listed in Appendix
\ref{app:fx_pairs_spread}.

In order to avoid overnight effects and any artifact due to the opening and
closing of the foreign exchange market, we systematically discard the first
ten and the last ten minutes of trading in a given week
\cite{Bouchaud_2004,my_paper_response_financial,Wang_2016_cross,large_prices_changes,spread_changes_affect}.
Therefore, we only consider trades of the same week from Sunday 19:10:00 to
Friday 16:50:00 New York local time. We will refer to this interval of time as
the ``market time".
