\abstract{
      In the past, due to the lack of foreign exchange data, an analysis
      considering large intervals of time was not possible. Nowadays, there is
      a large interest in foreign exchange market and because of the
      considerable amount of data, it is possible to access tick data for
      different years. We analyzed price response functions in the foreign
      exchange market for different years and different time scales. The price
      response functions show an increase to a maximum followed by a slowly
      decrease as the time lag grows, in trade time scale and in physical time
      scale for all the analyzed years. Further, we use a pip spread definition
      to group different foreign exchange pairs and analyze the impact of the
      spread in the price response functions. We found that large pip spreads
      have stronger impact in the response. The results of this paper were also
      seen in price responses functions in correlated financial markets.
\PACS{
      {89.65.Gh}{Econophysics} \and
      {89.75.-k}{Complex systems} \and
      {05.10.Gg}{Statistical physics}
     } % end of PACS codes
} %end of abstract