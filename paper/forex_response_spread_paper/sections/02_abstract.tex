\begin{abstract}
      We carry out a detailed large-scale data analysis of price response
      functions in the spot foreign exchange market for different years and
      different time scales. Such response functions provide quantitative
      information on the deviation from Markovian behavior. The price response
      functions show an increase to a maximum followed by a slow decrease as
      the time lag grows, in trade time scale and in physical time scale, for
      all analyzed years. Furthermore, we use a price increment point (pip)
      bid-ask spread definition to group different foreign exchange pairs and
      analyze the impact of the bid-ask spread in the price response functions.
      We find that large pip bid-ask spreads have a stronger impact on the
      response. This is similar to what has been found in stock markets.
\end{abstract} %end of abstract

\begin{keyword}
      Econophysics \sep Complex systems \sep Statistical physics \sep
      Price response function \sep bid-ask spread impact \sep
      Foreign exchange market
\end{keyword}