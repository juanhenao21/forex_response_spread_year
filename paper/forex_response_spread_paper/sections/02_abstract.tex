\abstract{
      We analyze price response functions in the foreign exchange market for
      different years and different time scales. Such response functions
      provide quantitative information on the deviation from Markovian
      behavior. The price response functions show an increase to a maximum
      followed by a slow decrease as the time lag grows, in trade time scale
      and in physical time scale, for all analyzed years.  Furthermore, we use
      a price increment point (pip) bid-ask spread definition to group
      different foreign exchange pairs and analyze the impact of the spread in
      the price response functions. We found that large pip spreads have
      stronger impact on the response. This is similar to what has been found
      in stock markets.
\PACS{
      {89.65.Gh}{Econophysics} \and
      {89.75.-k}{Complex systems} \and
      {05.10.Gg}{Statistical physics}
     } % end of PACS codes
} %end of abstract